\documentclass[12pt]{report}
\usepackage[utf8]{inputenc}

%\usepackage{ulsy}
% not needed with polyglossia
\usepackage[utf8]{inputenc}
\usepackage[T1]{fontenc}
\usepackage{dsfont}

%xe/lualatex
% \usepackage{polyglossia}
% \setdefaultlanguage{icelandic}


\usepackage{graphics,amsmath,amsfonts}
%\usepackage{theoremref}
\usepackage{amsbsy,amssymb}
\usepackage{amsthm}
\usepackage{fancyvrb}
\usepackage[a4paper]{geometry}
\usepackage{graphicx}
\usepackage{hyperref}
\usepackage{datatool}
\usepackage{float}
\usepackage[framemethod=tikz]{mdframed}
\usepackage{listingsutf8}
\usepackage{enumerate}
\usepackage{comment}
\usepackage{epstopdf}
\usepackage{caption}
\usepackage{subcaption}
\usepackage{titling}
\usepackage{tikz}
\usepackage[shortlabels]{enumitem}
\usepackage{mathtools}
\usepackage{tabu}

\usepackage{accents}

\usepackage{ stmaryrd }

%\usepackage[parfill]{parskip}

\setlength{\parskip}{8pt plus 1pt minus 1pt}
%Verdur ad vera her, sumir pakkar dependa a thetta.
% \usepackage[english]{babel}

\newcommand{\subtitle}[1]{%
  \posttitle{%
    \par\end{center}
    \begin{center}\large#1\end{center}
    \vskip0.5em}%
}
%viljum ekki númeraða kafla á dæmum


\newcommand{\runa}[1]{\left(#1 \right)_{n \geq 1}}
\newcommand{\limfty}{\underset{n\to\infty}{\lim}}
\newcommand{\exR}{\bar{\R}}
\newcommand{\esssup}{{\ess\sup}}
\newcommand{\essinf}{{\ess\inf}}
\newcommand{\mCh}{\mathds{1}}

\newcommand{\nonums}{\setcounter{secnumdepth}{-1}}

%flýtiskipanir
\newcommand{\e}{\emph}


%\newcommand{\R}{\Real}
%\newcommand{\C}{\Complex}
%\newcommand{\Z}{\Integer}
%\newcommand{\N}{\Natural}
%\newcommand{\Q}{\Rational}
\newcommand{\R}{\mathbb{R}}
\newcommand{\X}{\mathbb{X}}
\newcommand{\Y}{\mathbb{Y}}
\newcommand{\K}{\mathbb{K}}
\newcommand{\C}{\mathbb{C}}
\newcommand{\Con}{\mathcal{C}}
\newcommand{\Z}{\mathbb{Z}}
\newcommand{\N}{\mathbb{N}}
\newcommand{\Q}{\mathbb{Q}}
\newcommand{\f}{\frac}
\newcommand{\1}{\frac{1}}
\newcommand{\eps}{\f{\epsilon}}
\newcommand{\Lra}{\Leftrightarrow}
\newcommand{\Th}{\text{ þegar }}
\newcommand{\Ef}{\text{ ef }}
\newcommand{\Og}{\text{ og }}


\newcommand{\inner}[1]{\accentset{\circ}{#1}}
\newcommand{\eR}{\widetilde{\R}}

\newcommand{\sumninfty}[1]{\sum_{n = {#1}}^{\infty}}
\newcommand{\sumoinfty}{\sum_{n = 1}^{\infty}}
\newcommand{\summinfty}{\sum_{m = 1}^{\infty}}
\newcommand{\sumzinfty}{\sum_{n = 0}^{\infty}}

\newcommand{\com}[1]{\set{\text{#1}}}
\newcommand{\Com}[1]{\set{\text{Athsmd: \text{#1}}}}

\newcommand{\ub}[2]{\underbrace{#1}_{\text{#2}}}
\newcommand{\ubt}[2]{$\ub{\text{#1}}{#2}$}


\newenvironment{inum}{\begin{enumerate}[label=(\roman*).]}{\end{enumerate}}
\newenvironment{anum}{\begin{enumerate}[label=(\alph*).]}{\end{enumerate}}


\newcommand{\bcondef}{\left\{ \begin{array}{l l}}
\newcommand{\abs}[1]{{\left|#1\right|}}
\newcommand{\norm}[1]{{\left|\left|#1\right|\right|}}

\newcommand{\econdef}{\end{array} \right.}
\DeclarePairedDelimiter{\condef}{\bcondef}{\econdef}

\DeclarePairedDelimiter{\ceil}{\lceil}{\rceil}
\DeclarePairedDelimiter{\floor}{\lfloor}{\rfloor}
\DeclarePairedDelimiter{\set}{\{}{\}}
\DeclarePairedDelimiter{\braket}{\langle}{\rangle}

\DeclareMathOperator{\Ker}{Ker}
\DeclareMathOperator{\Img}{Img}
\DeclareMathOperator{\Rea}{Re}
\DeclareMathOperator{\Imaging}{Im}

\renewcommand{\Re}{\Rea}
\renewcommand{\Im}{\Imaging}
\DeclareMathOperator{\Span}{Span}

\newcommand{\sep}{\;|\;}

\newcommand{\fig}[2]{
\begin{figure}[H]
  \centering
  \includegraphics{#1}
  \caption{#2}
  \label{fig:#1}
\end{figure}
}



\renewcommand{\qedsymbol}{\textbf{Q.E.D}}

\mdfdefinestyle{theoremstyle}{%
roundcorner=5pt, %
%leftmargin=1pt, rightmargin=1pt, %
hidealllines=true, %
align = center, % align theenvironment itself (left, center, rigth)
nobreak=true,
}

\numberwithin{equation}{section}
\theoremstyle{plain}
\newtheorem*{setn}{Theorem}
\newtheorem{setn*}[equation]{Theorem}
\newtheorem*{hsetn}{lemma}

\theoremstyle{definition}
\newtheorem*{skgr}{Definition}
\newtheorem{skgr*}[equation]{Definition}
\newtheorem*{daemi}{Example}
\newtheorem{daemi*}[equation]{Example}
\newtheorem*{frumsenda}{Axiom}
\newtheorem*{lausn}{Lausn}


\theoremstyle{remark}
\newtheorem*{ath}{Remark}
\newtheorem*{innsk}{Innskot}

\surroundwithmdframed[style=theoremstyle,backgroundcolor = green!20]{skgr}
\surroundwithmdframed[style=theoremstyle,backgroundcolor=blue!10]{setn}
\surroundwithmdframed[style=theoremstyle,backgroundcolor = orange!20]{daemi}

\surroundwithmdframed[style=theoremstyle,backgroundcolor = green!20]{skgr*}
\surroundwithmdframed[style=theoremstyle,backgroundcolor=blue!10]{setn*}
\surroundwithmdframed[style=theoremstyle,backgroundcolor = orange!20]{daemi*}

\surroundwithmdframed[style=theoremstyle,backgroundcolor = red!20]{ath}



\lstset{  literate={á}{{\'a}}1
                  {ó}{{\'o}}1
                  {ú}{{\'u}}1
                  {ð}{{\dh}}1
                  {í}{{\'i}}1
                  {é}{{\'e}}1
                  {ö}{{\"o}}1
                  {þ}{{\th}}1
                  {æ}{{\ae}}1
                  {ý}{{\'y}}1
                  {Á}{{\'A}}1
                  {Ó}{{\'O}}1
                  {Ú}{{\'U}}1
                  {Ð}{{\DH}}1
                  {Í}{{\'I}}1
                  {É}{{\'E}}1
                  {Ö}{{\"O}}1
                  {Þ}{{\TH}}1
                  {Æ}{{\AE}}1
                  {Ý}{{\'Y}}1}


% Maður man víst betur stuff sem er skrifað
% í fonti sem erfitt er að lesa (t.d. sans serif)
%\renewcommand*{\familydefault}{\sfdefault}


\title{Logic in Computer Science}
\author{Matthías Páll Gissurarson} \date{Autumn 2015}


\newcommand{\dmu}{\text{d}\mu}
\newcommand{\cEf}{& \Ef}
\newcommand{\cP}{\mathcal{P}}
\newcommand{\cF}{\mathcal{F}}
\newcommand{\cB}{\mathcal{B}}
%\newcommand{\cS}{\mathcal{S}}
\newcommand{\cM}{\mathcal{M}}
\newcommand{\cN}{\mathcal{N}}
\newcommand{\cG}{\mathcal{G}}
\newcommand{\cH}{\mathcal{H}}
\newcommand{\cL}{\mathcal{L}}
\newcommand{\cS}{\mathcal{S}}
\newcommand{\cA}{\mathcal{A}}
\newcommand{\U}{\mathfrak{U}}
\newcommand{\dMu}{\text{d}\mu}
\newcommand{\aabs}[1]{\abs{\abs{#1}}}
\newcommand{\invf}{f^{-1}}
\newcommand{\ess}{\text{ess}\:}
\newcommand{\ind}[1]{\mathds{1}_{#1}}
\DeclareMathOperator{\Real}{Re}
\DeclareMathOperator{\Imag}{Im}
\renewcommand{\Re}{\Real}
\renewcommand{\Im}{\Imag}


\newcommand{\txf}[2]{\f{\text{#1}}{\text{#2}}}

\begin{document}

\maketitle

\tableofcontents

\chapter{2015-09-01}

\url{www.cse.chalmers.se/edu/course/DAT060}

What is logic about?
\begin{itemize}
  \item Rules for correct reasoning.
  \item Foundation of mathematics.
  \item Mathematics about mathematics (metamathematics)
\end{itemize}

History:
\begin{itemize}
    \item Babylonian mathematics $\sim$ 2000 BC.
      Had no notion of proof, but gave very general examples. Could
      calculate $\sqrt{2}$ to 7 decimals.
    \item Pythagoras 580-495 BC.   Were shocked by that
      the $\sqrt{2}$ was irrational.
    \item \textbf{Aristotle} 384-322 BC
      Called his rules for \textbf{Syllogism}, and had \textbf{24} of them.

      Example: 
        All dogs have four legs
        Carlo is a dog.
        => Carlo has four legs
    \item Euclid $\sim$ 300 BC.  Geometry. Wrote Euclids elements, which
      used axioms. Used for a long time in schools.
    \item Archimedes 287-212 BC.

    \item Islamic Logic:  Avicerna (ibn-suna) 980-1037 AD.
     
    \item Medival logic: 1200-1600 BC.

\end{itemize}

Scientific revolution:
\begin{itemize}
\item Galileo 1564-1642. The first one to conduct proper experiments. Science, not logic nor mathemathics.
\item Descartes 1596-1650. Started an intellectual revolution. Was Both a
  philosopher and a mathematician. Rationalism, ``Cogito ergo sum''.
He wanted to write everything we know from the start.
\item \textbf{Leibniz} 1646-1716. Also invented differential calculus.
 Wanted to construct a langugae in which any argument could be represented
 and checked to be correct or not.
\item Newton 1646-1727.
\end{itemize}

Newton, Leibniz and Descartes were all convinced that god existed, and use that
as a basis for some of their proofs, even though it is unproven.

Empiricism:
\begin{itemize}
    \item Hume 1711 - 1776. Was a bit secular, and was interested in what it was
      possible to know.
    \item Boole 1815 - 1864 Laws of thought.
      Propositional logic. $\wedge \vee, \rightarrow, \lnot$
    \item \textbf{Gottlieb Frege} 1848 - 1925
      Wrote a very important text about logic Begrüffnschift (1879)
      Predicate logic, proposistional logic plus quantifiers: $\forall,\exists$

      Grundgesetsse 1893, 1902
      If P(x) is a property, then we can form the set $\set{ x | P(x)}$

      \item Bertrand Russel. Found the Russel paradox, i.e.:
        According to Frege, you can form the set 
        \[ A = \set{x | x \not\in x }\]
        Assume that $A \in A$ (1)
         According to the definition we then get that $A \not\in A$ which
          contradicts the assumption which hence must be false, i.e. $A \not\in
          A$, but then (2) follow, by the definition of A, that $A \in A$.
        \item  Cantor $\sim$ 1870 Set Theory.
          Potential infinity $0,1,2, \dotsc, n, n+1, \dotsc$
          Actual infinity:
          $\set{0,1,2, \dotsc}$
        \item Zemelo 1908 Given a set $G$, then we can form $\set{x \in G | P(x)}$
        \item Brouwer
          All mathematical objects must be constructed by us.
          A proposition is true if and only if we can prove it (complete).
          $A \vee \lnot A$.
\end{itemize}

\begin{itemize}
\item Constructivism: Only things we can construct exists.
\item Platonism: Objects exists independent of us.
\item Formalism: Things can only be derived.
\end{itemize}


\chapter{2015-09-04}

Missed


\chapter{2015-09-08}

\section{Inductive definitions}

The set $N$ of natural numbers is inductively defined by
\begin{enumerate}[(i)]
\item 0 is a natural number
\item if $n$ is a natrual number, then $\text{succ}(n)$ is a natural number.
\end{enumerate}

Expressed by rules:
\begin{enumerate}[(i)]
\item $0 \in \N$
\item $\f{n \in \N}{\text{succ}(n) \in \N}$
\end{enumerate}

Example of a recursively defined function, $n!$

\[\bcondef \text{fac}(0) & = 1 \\ \text{fac(succ($n$))} & \text{succ}(n) \cdot
  \text{fac}(n) \econdef \]


\end{document}
