\documentclass[12pt]{report}
\usepackage[utf8]{inputenc}

%\usepackage{ulsy}
\input{header.tex}

% Maður man víst betur stuff sem er skrifað
% í fonti sem erfitt er að lesa (t.d. sans serif)
%\renewcommand*{\familydefault}{\sfdefault}


\title{Logic in Computer Science}
\author{Matthías Páll Gissurarson} \date{Autumn 2015}


\newcommand{\dmu}{\text{d}\mu}
\newcommand{\cEf}{& \Ef}
\newcommand{\cP}{\mathcal{P}}
\newcommand{\cF}{\mathcal{F}}
\newcommand{\cB}{\mathcal{B}}
%\newcommand{\cS}{\mathcal{S}}
\newcommand{\cM}{\mathcal{M}}
\newcommand{\cN}{\mathcal{N}}
\newcommand{\cG}{\mathcal{G}}
\newcommand{\cH}{\mathcal{H}}
\newcommand{\cL}{\mathcal{L}}
\newcommand{\cS}{\mathcal{S}}
\newcommand{\cA}{\mathcal{A}}
\newcommand{\U}{\mathfrak{U}}
\newcommand{\dMu}{\text{d}\mu}
\newcommand{\aabs}[1]{\abs{\abs{#1}}}
\newcommand{\invf}{f^{-1}}
\newcommand{\ess}{\text{ess}\:}
\newcommand{\ind}[1]{\mathds{1}_{#1}}
\DeclareMathOperator{\Real}{Re}
\DeclareMathOperator{\Imag}{Im}
\renewcommand{\Re}{\Real}
\renewcommand{\Im}{\Imag}


\newcommand{\txf}[2]{\f{\text{#1}}{\text{#2}}}

\begin{document}

\maketitle

\tableofcontents

\chapter{2015-09-01}

\url{www.cse.chalmers.se/edu/course/DAT060}

What is logic about?
\begin{itemize}
  \item Rules for correct reasoning.
  \item Foundation of mathematics.
  \item Mathematics about mathematics (metamathematics)
\end{itemize}

History:
\begin{itemize}
    \item Babylonian mathematics $\sim$ 2000 BC.
      Had no notion of proof, but gave very general examples. Could
      calculate $\sqrt{2}$ to 7 decimals.
    \item Pythagoras 580-495 BC.   Were shocked by that
      the $\sqrt{2}$ was irrational.
    \item \textbf{Aristotle} 384-322 BC
      Called his rules for \textbf{Syllogism}, and had \textbf{24} of them.

      Example: 
        All dogs have four legs
        Carlo is a dog.
        => Carlo has four legs
    \item Euclid $\sim$ 300 BC.  Geometry. Wrote Euclids elements, which
      used axioms. Used for a long time in schools.
    \item Archimedes 287-212 BC.

    \item Islamic Logic:  Avicerna (ibn-suna) 980-1037 AD.
     
    \item Medival logic: 1200-1600 BC.

\end{itemize}

Scientific revolution:
\begin{itemize}
\item Galileo 1564-1642. The first one to conduct proper experiments. Science, not logic nor mathemathics.
\item Descartes 1596-1650. Started an intellectual revolution. Was Both a
  philosopher and a mathematician. Rationalism, ``Cogito ergo sum''.
He wanted to write everything we know from the start.
\item \textbf{Leibniz} 1646-1716. Also invented differential calculus.
 Wanted to construct a langugae in which any argument could be represented
 and checked to be correct or not.
\item Newton 1646-1727.
\end{itemize}

Newton, Leibniz and Descartes were all convinced that god existed, and use that
as a basis for some of their proofs, even though it is unproven.

Empiricism:
\begin{itemize}
    \item Hume 1711 - 1776. Was a bit secular, and was interested in what it was
      possible to know.
    \item Boole 1815 - 1864 Laws of thought.
      Propositional logic. $\wedge \vee, \rightarrow, \lnot$
    \item \textbf{Gottlieb Frege} 1848 - 1925
      Wrote a very important text about logic Begrüffnschift (1879)
      Predicate logic, proposistional logic plus quantifiers: $\forall,\exists$

      Grundgesetsse 1893, 1902
      If P(x) is a property, then we can form the set $\set{ x | P(x)}$

      \item Bertrand Russel. Found the Russel paradox, i.e.:
        According to Frege, you can form the set 
        \[ A = \set{x | x \not\in x }\]
        Assume that $A \in A$ (1)
         According to the definition we then get that $A \not\in A$ which
          contradicts the assumption which hence must be false, i.e. $A \not\in
          A$, but then (2) follow, by the definition of A, that $A \in A$.
        \item  Cantor $\sim$ 1870 Set Theory.
          Potential infinity $0,1,2, \dotsc, n, n+1, \dotsc$
          Actual infinity:
          $\set{0,1,2, \dotsc}$
        \item Zemelo 1908 Given a set $G$, then we can form $\set{x \in G | P(x)}$
        \item Brouwer
          All mathematical objects must be constructed by us.
          A proposition is true if and only if we can prove it (complete).
          $A \vee \lnot A$.
\end{itemize}

\begin{itemize}
\item Constructivism: Only things we can construct exists.
\item Platonism: Objects exists independent of us.
\item Formalism: Things can only be derived.
\end{itemize}


\chapter{2015-09-04}

Missed


\chapter{2015-09-08}

\section{Inductive definitions}

The set $N$ of natural numbers is inductively defined by
\begin{enumerate}[(i)]
\item 0 is a natural number
\item if $n$ is a natrual number, then $\text{succ}(n)$ is a natural number.
\end{enumerate}

Expressed by rules:
\begin{enumerate}[(i)]
\item $0 \in \N$
\item $\f{n \in \N}{\text{succ}(n) \in \N}$
\end{enumerate}

Example of a recursively defined function, $n!$

\[\bcondef \text{fac}(0) & = 1 \\ \text{fac(succ($n$))} & \text{succ}(n) \cdot
  \text{fac}(n) \econdef \]


\end{document}
