\documentclass[12pt]{report}
\usepackage[utf8]{inputenc}

%\usepackage{ulsy}
\input{header.tex}

% Maður man víst betur stuff sem er skrifað
% í fonti sem erfitt er að lesa (t.d. sans serif)
%\renewcommand*{\familydefault}{\sfdefault}


\title{Logic in Computer Science}
\author{Matthías Páll Gissurarson} \date{Autumn 2015}


\newcommand{\dmu}{\text{d}\mu}
\newcommand{\cEf}{& \Ef}
\newcommand{\cP}{\mathcal{P}}
\newcommand{\cF}{\mathcal{F}}
\newcommand{\cB}{\mathcal{B}}
%\newcommand{\cS}{\mathcal{S}}
\newcommand{\cM}{\mathcal{M}}
\newcommand{\cN}{\mathcal{N}}
\newcommand{\cG}{\mathcal{G}}
\newcommand{\cH}{\mathcal{H}}
\newcommand{\cL}{\mathcal{L}}
\newcommand{\cS}{\mathcal{S}}
\newcommand{\cA}{\mathcal{A}}
\newcommand{\U}{\mathfrak{U}}
\newcommand{\dMu}{\text{d}\mu}
\newcommand{\aabs}[1]{\abs{\abs{#1}}}
\newcommand{\invf}{f^{-1}}
\newcommand{\ess}{\text{ess}\:}
\newcommand{\ind}[1]{\mathds{1}_{#1}}
\DeclareMathOperator{\Real}{Re}
\DeclareMathOperator{\Imag}{Im}
\renewcommand{\Re}{\Real}
\renewcommand{\Im}{\Imag}


\newcommand{\txf}[2]{\f{\text{#1}}{\text{#2}}}
\newcommand{\If}{\text{if }}

\begin{document}

\maketitle

\tableofcontents

\chapter{2015-09-01}

\url{www.cse.chalmers.se/edu/course/DAT060}

What is logic about?
\begin{itemize}
  \item Rules for correct reasoning.
  \item Foundation of mathematics.
  \item Mathematics about mathematics (metamathematics)
\end{itemize}

History:
\begin{itemize}
    \item Babylonian mathematics $\sim$ 2000 BC.
      Had no notion of proof, but gave very general examples. Could
      calculate $\sqrt{2}$ to 7 decimals.
    \item Pythagoras 580-495 BC.   Were shocked by that
      the $\sqrt{2}$ was irrational.
    \item \textbf{Aristotle} 384-322 BC
      Called his rules for \textbf{Syllogism}, and had \textbf{24} of them.

      Example: 
        All dogs have four legs
        Carlo is a dog.
        => Carlo has four legs
    \item Euclid $\sim$ 300 BC.  Geometry. Wrote Euclids elements, which
      used axioms. Used for a long time in schools.
    \item Archimedes 287-212 BC.

    \item Islamic Logic:  Avicerna (ibn-suna) 980-1037 AD.
     
    \item Medival logic: 1200-1600 BC.

\end{itemize}

Scientific revolution:
\begin{itemize}
\item Galileo 1564-1642. The first one to conduct proper experiments. Science, not logic nor mathemathics.
\item Descartes 1596-1650. Started an intellectual revolution. Was Both a
  philosopher and a mathematician. Rationalism, ``Cogito ergo sum''.
He wanted to write everything we know from the start.
\item \textbf{Leibniz} 1646-1716. Also invented differential calculus.
 Wanted to construct a langugae in which any argument could be represented
 and checked to be correct or not.
\item Newton 1646-1727.
\end{itemize}

Newton, Leibniz and Descartes were all convinced that god existed, and use that
as a basis for some of their proofs, even though it is unproven.

Empiricism:
\begin{itemize}
    \item Hume 1711 - 1776. Was a bit secular, and was interested in what it was
      possible to know.
    \item Boole 1815 - 1864 Laws of thought.
      Propositional logic. $\wedge \vee, \rightarrow, \lnot$
    \item \textbf{Gottlieb Frege} 1848 - 1925
      Wrote a very important text about logic Begrüffnschift (1879)
      Predicate logic, proposistional logic plus quantifiers: $\forall,\exists$

      Grundgesetsse 1893, 1902
      If P(x) is a property, then we can form the set $\set{ x | P(x)}$

      \item Bertrand Russel. Found the Russel paradox, i.e.:
        According to Frege, you can form the set 
        \[ A = \set{x | x \not\in x }\]
        Assume that $A \in A$ (1)
         According to the definition we then get that $A \not\in A$ which
          contradicts the assumption which hence must be false, i.e. $A \not\in
          A$, but then (2) follow, by the definition of A, that $A \in A$.
        \item  Cantor $\sim$ 1870 Set Theory.
          Potential infinity $0,1,2, \dotsc, n, n+1, \dotsc$
          Actual infinity:
          $\set{0,1,2, \dotsc}$
        \item Zemelo 1908 Given a set $G$, then we can form $\set{x \in G | P(x)}$
        \item Brouwer
          All mathematical objects must be constructed by us.
          A proposition is true if and only if we can prove it (complete).
          $A \vee \lnot A$.
\end{itemize}

\begin{itemize}
\item Constructivism: Only things we can construct exists.
\item Platonism: Objects exists independent of us.
\item Formalism: Things can only be derived.
\end{itemize}


\chapter{2015-09-04}

Missed


\begin{skgr}[$\vdash$ (derives)]
  Syntatic implication.
  If $S \vdash \psi$, it means that $\psi$ can be
  derived from the formulas in $S$.
\end{skgr}

\begin{skgr}[$\models$ (models)]
Semantic implication.
$A \models B$ means that $B$ is true in every model in which $A$ is true.
  
\end{skgr}

\chapter{2015-09-08}

\section{Inductive definitions}

The set $N$ of natural numbers is inductively defined by
\begin{enumerate}[(i)]
\item 0 is a natural number
\item if $n$ is a natrual number, then $\text{succ}(n)$ is a natural number.
\end{enumerate}

Expressed by rules:
\begin{enumerate}[(i)]
\item $0 \in \N$
\item $\f{n \in \N}{\text{succ}(n) \in \N}$
\end{enumerate}

Example of a recursively defined function, $n!$

\[\bcondef \text{fac}(0) & = 1 \\ \text{fac(succ($n$))} & = \text{succ}(n) \cdot
  \text{fac}(n) \econdef \]


The set $\cF$ of proposistional formulas is inductively defined by
\begin{enumerate}[(i)]
\item  atoms, including $\bot$, are formulas
\item if $\emptyset$ and $\psi$ are formulas, so are
$(\emptyset \wedge \psi),
 (\emptyset \vee \psi),
 (\emptyset \to \psi)$
and $(\lnot \emptyset)$
\end{enumerate}

\begin{daemi}
  Define a function par which computes the number of parentheses
in a formula. Done by recursion.

\begin{enumerate}[(i)]
\item  $\text{par}(\emptyset) = 0$ when $\emptyset$ is an atom.
\item
  \begin{gather*}
   \text{par}((\emptyset \wedge \psi)) = \text{par}(\emptyset) +
    \text{par}(\psi) + 2 \\
   \text{par}((\emptyset \vee \psi)) = \text{par}(\emptyset) +
    \text{par}(\psi) + 2 \\
    \vdots\\
    \text{par}((\lnot \emptyset)) = \text{par}(\emptyset) + 2
  \end{gather*}
\end{enumerate}
\end{daemi}

Truth tables

\begin{tabular*}{0.5\linewidth}{c c | c | c | c | c}
  $\emptyset$ & $\psi$ & $\emptyset \wedge \psi$ & $\emptyset \vee \psi$ &
                                                                           $\emptyset
                                                                           \to
                                                                           \psi$
  & $\lnot \emptyset$\\
\hline
T & T &T & T & T & F \\
T & F & F & T & F & F \\
F & T & F & T & T & T \\
F & F & F & F & T & T \\
\end{tabular*}

Two kind of semantics
\begin{enumerate}[(i)]
\item  in terms of truth values ``The meaning of a formula is its truth value''
  (Frege)
\item In terms of proofs. Constructive semantics.
\end{enumerate}

\begin{skgr}[1.28, p 37]

A \textbf{valuation} $v$ is a function from the set of atoms to the set of truth
values
$v: \set{p,q,r, p_1, p_2, p_3, \dotsc} \to \set{T,F}$
$v$ can be extended to the set of proposistional formulas by recursion.

\begin{enumerate}[(i)]
\item $v$ is already defined on the atoms ($v(\bot) = F$)
\item
    \[v(\emptyset \wedge \psi) = \bcondef T & \If v(\psi) = v(\emptyset) = T \\ F
    & \text{Otherwise} \econdef \\\]
    \[v(\emptyset \vee \psi) = \bcondef T & \If v(\psi) = T \text{ or } v(\emptyset) = T \\ F
    & \text{Otherwise} \econdef \\\]
    \[v(\emptyset \to \psi) = \bcondef F & \If v(\emptyset) = T \text{ and } v(\psi) = F \\ T
    & \text{Otherwise} \econdef \\\]
    \[v(\not \emptyset ) = \bcondef T & \If v(\emptyset) = F \\ F
    & \text{Otherwise} \econdef \\\]

\end{enumerate}
\end{skgr}

\begin{skgr}[1.34, p 46]

If for all valuations in which all $\emptyset_1,\dotsc, \emptyset_n$ evaluates
to $T$, also $\psi$ evaluates to $T$ then \[\emptyset_1, \dotsc, \emptyset_n
\models \psi\]
holds.
\end{skgr}

\begin{daemi}
  $p \wedge q \models p$ holds.
Nota $p \wedge q \vdash$ with a completely different argument, namely the
derivation
\begin{enumerate}[1.]
\item $p \wedge q$ premise
\item \(p\)     $\wedge e_1, 1$
\end{enumerate}

\end{daemi}

\begin{daemi}
  \[ p,q \models p \wedge q \]
\end{daemi}


\begin{setn}[Soundness (1.35)]

\[ \emptyset_1, \dotsc, \emptyset_n \vdash \psi \Rightarrow \emptyset_1,
  \dotsc, \emptyset_n \models \psi\]
\end{setn}
\begin{proof}
Proof by induction on the derivation
\[ \f{\shortstack{\(\vdots \;\;\; \vdots\) \\ \(\psi_1 \;\;\;  \psi_2\)}}{\psi_1 \wedge \psi_2} \wedge i \]
\[\f{\shortstack{\(\vdots\) \\ \(\psi_1 \wedge  \psi_2\)}}{\psi_1} \wedge e_i\]
\[\f{\shortstack{\(\psi_1\)\\ \(\vdots\)\\ \(\psi_2\)}}{\psi_1 \to \psi_2} \to i\]
\end{proof}

\begin{setn}[1.4.4 p 49. Completeness]

\[\emptyset_1, \dotsc, \emptyset_n \models \psi \to \emptyset_1, \dotsc,
  \emptyset_n \vdash \psi\]
Proof by kalmar 1930.
The construction of Kalmar's derivation is exponential in the number of 
atoms of the formulas $\sim 2^n$, where $n$ is the number of atoms.
\end{setn}

\begin{skgr}
  If $\models \emptyset$ hodls, we say that $\emptyset$ is a \textbf{tautology}
\end{skgr}
$\emptyset$ is true for all valuations

$\llbracket \emptyset \rrbracket_r = T$ for all $v$

\[\emptyset_1, \dotsc, \emptyset_n \models \psi \Leftrightarrow \models
  (\emptyset_1 \wedge \dotsb \wedge \emptyset_n) \to \psi \]

\section{Constructive semantics}

Brouwer, Heyting, Kolmogorov. 1930.

A proposistion is defined by laying down what counts as a proof of it.

\begin{tabular*}{1.0\linewidth}{c | l | l}
  To prove& we must& Type\\
  \hline
  \(\emptyset \wedge \psi\) & prove \(\emptyset\) and prove \(\psi\) & \(A \times B\)\\
  \hline
  \(\emptyset \vee \psi\) & prove \(\emptyset\) or prove \(\psi\) & \(A + B\)\\
  \hline
  \(\emptyset \to \psi\) &  give a method which & \(A \to B\)\\
  & to each proof of \(\emptyset\) gives a proof of \(\psi\)  & \\
  \hline
  \(\bot\) & nothing counts as a proof of $\bot$ & \(\bot\) \\
\end{tabular*}

Bishop 1967. Foundations of constructive analysis, a great work in constructive
semantics.

Thierry Coquand was given a prize for his work in logic.



\begin{align*}
  \wedge i: \;\;\; \f{\emptyset \;\;\; \psi}{\emptyset \wedge \psi} 
    && \text{x-induction}: \;\;\; \f{a \in A \;\;\; b \in B}{\langle a,b \rangle \in A \times B} \times i\\
\wedge e_1: \;\;\; \f{\emptyset \wedge \psi}{\emptyset} \\
&& \text{x-elimination}: \;\;\; \f{c \in A \times B}{ \text{fst}(c) \in A} \times e_1\\
\wedge e_2: \;\;\; \f{\emptyset \wedge \psi}{\psi} \\
&& \text{x-elimination}: \;\;\; \f{c \in A \times B}{ \text{snd}(c) \in B} \times e_2
\end{align*}

Where
 $\text{fst}(\langle  a, b \rangle) = a$  and
 $\text{snd}(\langle  a, b \rangle) = b$ .

Curry 1958. - noticed implications\\
Howard 1969 - extended to predicate logic.\\
Per Marten-Löf 1970. Type theory.

\end{document}
